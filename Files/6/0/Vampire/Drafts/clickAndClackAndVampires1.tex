\documentclass{article}
\title{Click and Clack and Vampires}
\author{D. Michael Parrish}
\date{2021-05-23}
\begin{document}
\maketitle

\begin{center}
\texttt{
Are you sure you want to delete "term\_paper\_final?"\\
Click "Y" for "yes" or "N" for "no:"\\
\phantom{M}\\
YN
}
\end{center}
\phantom{M} \hfill ---OS WTF [1]

\begin{abstract}
\noindent
A method is described by which a tolerance controlling the
computer interpretation of pointing device (such as a mouse)
input may be investigated by individual users in an organic
manner.
The intended usage of the method is that each user may apply
the knowledge gained from the investigation to the setting of
one or more tolerances.
Computer subroutines may apply these tolerances to
the interpretation of each user's input.
The goal is that user input is interpreted in a manner
consistent with the user's intent.
\end{abstract}

\subsection*{Keywords}

customization, Human-computer interface, mouse, personal computing

\subsection*{Article History}

2021-05-23 First Complete Draft\\
2021-05-12 Partial Draft (description of method)\\

\subsection*{License}

Copyright 2021 D. Michael Parrish\\
This article is licensed to all under the terms of CC BY-SA 4.0,\\
https://creativecommons.org/licenses/by-sa/4.0/

\section*{Prologue}

Virginia was in a hurry.
She furiously clicked the file icons to select which ones she
wanted to attach to an urgent email response.

That's what she thought she was doing.
What actually happend was that her computer interpreted the
mouse input as
an instruction to move a directory containing one million files
into another directory.

Virgina felt personally betrayed by her personal computer.

\subsection*{Definition}

\textbf{vampire}: a trap door on a stage [2, 3]

\section*{Introduction}

Herein, we raise issue with a certain aspect of applications of
human-computer interfaces---namely pointing devices---and
present a method which could be part of a solution.

The issue and solution have to do with tolerances surrounding
the user's precision in specifying points.

Below, we present background information,
an organic method for estimating tolerance(s),
and a particular application of the method.

\section*{Background}

Several widely-used graphical user interfaces (GUI's) include a
facility for specifying a point.
Such a specification might be associated with the intent to
select an object (e.g., a file, a ``No'' button, or a menu) or
to create a new one (e.g., whatever is associated with a
particular menu item).

Some of these GUI's have a user-controlled parameter---call it
\emph{tolerance}---that defines ``close enough.'' In other
words---suppose the user's pointing device is a mouse---between
the button-down and button-up events, the mouse may move, and so
there are at least two points associated with the \emph{click}
(botton-down followed by button-up events).
If the two points are close enough, they are taken to represent
a single point. If the two points are too far apart, the point
might be ignored.

However, there may be something intolerable about the tolerance:
the manner in which the user might gain knowledge sufficient to
appropriately specify it.
Typically, the tolerance is specified by a \emph{number}.
How is Virginia supposed to know which number corresponds to
the steadyness of her hand?
One method by which she might come to such knowledge is
presented as follows.

\section*{Method}

The user moves the mouse quickly, guiding the mouse cursor to an
arbitrary location, then clicks. This process is repeated many
times.

Each time the mouse-button state changes, the mouse position is
logged.
There are two positions for each click (one for button-down, one
for button-up), as well as the \emph{duration} of the click.

When enough repetitions have been performed, each of the logged
pairs of positions are differenced.
Each click $k$ might be represented

$$(\Delta t, \Delta x,  \Delta y)_k$$

Each of these differences represent the user's accuracy in
picking a single point;
$(\Delta t, 0, 0)$ would correspond to exactness.

The differences are organized into three collections: one for
each direction $x$ and $y$, and one for duration.

Post-processing and statistics are computed on the three
collections.
Post-processing steps might include taking the absolute value,
then replacing all the zero differences with the next smallest
value.
Post-processing might also include casting out the more extreme
absolute differences.

The statistics establish thresholds of tolerance in each
direction for estimating the user's intent when clicking: if the
differences in position between button-down and button-up fall
within the tolerance, the corresponding action might be
interpreted as the user specifying a single point.
If the differences in position are greater, but the
\emph{duration} $\Delta t$ is also large, the event might be
interpreted as a click-and-drag operation.
Finally, if the differences in position exceed the tolerance,
and $\Delta t$ is short (as when the user's elbow is bumped
mid-click), the event might be interpreted as noise, and
therefore to be ignored.

Post-processed results are presented to the user.

\section*{Anecdotes}

%%%%%%%%%%%%%%%%%%%%%%%%%%%%%%%%%%%%%%%%%%%%%%%%%%%%%%%%%%%%%%%%
As an example, the author conducted an experiment in which
the subject (the author) entered several mouse clicks, moving the
mouse quickly between clicks, and attempting to stop during each
click.

The mouse position was captured at each button-down and
button-up event.
The difference between the mouse position at button-down
and at button-up was computed for each click.

The resultant mouse movements were presented to the subject
(see the accompanying Plate 1 [4]).
The subject was free to use the information presented in order
to select a tolerance for future clicks.

\subsection*{Software}

The above experiment was conducted using software[6] running in
R[7] version 4.0.3.

\section*{Discussion}

Such a method might be offered to the user upon installation of
new software.
The method might also be made available in a ``Customize'' menu.

The anecdotes provided in the correspondingly-named section do
not include consideration of the time dimension.
We leave such consideration to future work, and perhaps the work
of others.

\section*{Conclusion}

In the author's opinion, personal computer software is too
often impersonal.
That is, facilities for tailoring the user experience to to the
user are lacking.
The present paper has presented a small part of what might be
a larger effort toward ``making the computer `personal again{'}{''}[5].

\section*{References}

[1] OS WTF (a fictitious operating system whose user-prompts may
be all too familiar.)

\phantom{M}

\noindent
[2] The Free Dictionary (2021-05-12) ``vampire.'' therein attributed to 
HarperCollins (2005) Collins Discovery Encyclopedia, 1 ed.
https://encyclopedia2.\allowbreak thefreedictionary.com/vampire

\phantom{M}

\noindent
[3] Wiki User (2013-12-13) ``Why is a stage trap door called a
vampire?'' https:/\slash
www.\allowbreak answers.\allowbreak com\slash
Q/Why\_is\_a\_stage\_trap\_door\_called\_a\_vampire

\phantom{M}

\noindent
[4] Parrish DM (2021-05-19) ``Test Results''
https:/\slash github.com\slash dmparrishphd\slash click\slash
blob\slash master\slash Files\slash 6/0\slash Vampire\slash
Drafts\slash clickAndClackAnd\allowbreak Vampires\_\\
Plate1.pdf

\phantom{M}

\noindent
%%%%%%%%%%%%%%%%%%%%%%%%%%%%%%%%%%%%%%%%%%%%%%%%%%%%%%%%%%%%%%%%
[5] HP (2008-06-10) ``HP Bridges Work, Play, Home and
Entertainment with Global Personal Systems Product Launch.''
https:/\slash www8.hp.com\slash us/en\slash hp-news\slash
press-release.html?\allowbreak id=169893

\phantom{M}

\noindent
[6] Parrish DM (2021) clickAndClackAndVampires.R
https:/\slash github.com\slash\\
dmparrishphd\slash click\slash
blob\slash master\slash Files\slash 6/0\slash Vampire/R\slash
clickAndClackAnd\\
Vampires.R

\phantom{M}

\noindent
[7] The R Foundation for Statistical Computing (2020-10-10) R v4.0.3
https:/\slash www.\allowbreak r-project.\allowbreak org/

\end{document}